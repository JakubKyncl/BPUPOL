%%%  Ukázkový text a dokumentace stylu pro text závěrečné (bakalářské a
%%%  diplomové) práce na KI PřF UP v Olomouci
%%%  Copyright (C) 2012 Martin Rotter, <rotter.martinos@gmail.com>
%%%  Copyright (C) 2014 Jan Outrata, <jan.outrata@upol.cz>


%%  Pro získání PDF souboru dokumentu je třeba tento zdrojový text v
%%  LaTeXu přeložit (dvakrát) programem pdfLaTeX.

%%  V případě použití programu BibLaTeX pro tvorbu seznamu literatury
%%  je poté ještě třeba spustit program Biber s parametrem jméno
%%  souboru zdrojového textu bez přípony a následně opět (dvakrát)
%%  přeložit zdrojový text programem pdfLaTeX.

%%  Postup získání Postscriptového souboru je popsán v dokumentaci.


%%  Třída dokumentu implementující styl pro závěrečnou práci. Vybrané
%%  nepovinné parametry (ostatní v dokumentaci):

%%  'master' pro sazbu diplomové práce, jinak se sází bakalářská práce

%%  'program=kód' pro Váš studijní program/obor (specializaci), kódy
%%  pro diplomovou práci 'infoi' pro Informatiku (Obecná informatika),
%%  'infui' pro Informatiku (Umělá inteligence), 'ainfpst' pro
%%  Aplikovanou informatiku (Počítačové systémy a technologie), 'uinf'
%%  pro Učitelství informatiky pro střední školy, 'binf' pro
%%  Bioinformatiku, 'inf' pro Informatiku (bez specializací) a 'ainf'
%%  pro Aplikovanou informatiku (bez specializací), jinak je výchozí
%%  ainfvs pro Aplikovanou informatiku (Vývoj software), a pro
%%  bakalářskou práci 'infoi' pro Informatiku (Obecná informatika),
%%  'itp' pro Informační technologie v prezenční formě, 'itk' pro
%%  Informační technologie v kombinované formě, 'infv' pro Informatiku
%%  pro vzdělávání, 'binf' pro Bioinfomatiku, 'inf' pro Informatiku
%%  (bez specializací), 'ainfp' pro Aplikovanou informatiku (bez
%%  specializací) v prezenční formě, 'ainfk' pro Aplikovanou
%%  informatiku (bez specializací) v kombinované formě, jinak je
%%  výchozí infpvs pro Informatiku (Programování a vývoj software)

%%  'printversion' pro sazbu verze pro tisk (nebarevné logo a odkazy,
%%  odkazy s uvedením adresy za odkazem, ne odkazy do rejstříku),
%%  jinak verze pro prohlížeč

%%  'biblatex' pro zapnutí podpory pro sazbu bibliografie pomocí
%%  BibLaTeXu, jinak je výchozí sazba v prostředí thebibliography

%%  'language=jazyk' pro jazyk práce, jazyky english pro anglický,
%%  slovak pro slovenský, jinak je výchozí czech pro český

%%  'font=sans' pro bezpatkový font (Iwona Light), jinak je výchozí
%%  serif pro patkový (Latin Modern)

%%  'figures, tables, theorems a sourcecodes' pro sazbu seznamu
%%  obrázků, tabulek, vět a zdrojových kódů, jinak při =false se
%%  nesází (u theorems a sourcecodes výchozí)

\documentclass[
%  master,
program=itp,
%  printversion,
  biblatex,
%  language=english,
%  font=sans,
  figures=false,
%  tables=false,
%  theorems,
%  sourcecodes,
  glossaries,
  index
]{kidiplom}

%% Informace pro úvodní strany. V jazyku práce (pokud není v komentáři
%% uvedeno česky) a anglicky. Uveďte všechny, u kterých není v
%% komentáři uvedeno, že jsou volitelné. Při neuvedení se použijí
%% výchozí texty. Text pro jiný než nastavený jazyk práce (nepovinným
%% parametrem language makra \documentclass, výchozí český) se zadává
%% použitím makra s uvedením jazyka jako nepovinného parametru.

%% Název práce, česky a anglicky. Měl by se vysázet na jeden řádek.
\title{Automatizace domácího chovu pomocí Arduina}
\title[english]{Arduino based automatization for breeding}

%% Volitelný podnázev práce, česky a anglicky. Měl by se vysázet na
%% jeden řádek. Výchozí je prázdný.


%% Jméno autora práce. Makro nemá nepovinný parametr pro uvedení
%% jazyka.
\author{Jakub Kyncl}

%% Jméno vedoucího práce (včetně titulů). Makro nemá nepovinný
%% parametr pro uvedení jazyka.
\supervisor{RNDr. Eduard Bartl, Ph.D.}

%% Volitelný rok odevzdání práce. Výchozí je aktuální (kalendářní)
%% rok. Makro nemá nepovinný parametr pro uvedení jazyka.
%\yearofsubmit{\the\year}

%% Anotace práce, včetně anglické (obvykle překlad z jazyka
%% práce). Jeden odstavec!
\annotation{Ukázkový text závěrečné práce na Katedře informatiky
  Přírodovědecké fakulty Univerzity Palackého v Olomouci, který je
  zároveň dokumentací stylu pro text práce v \LaTeX{}u. Zdrojový text
  v \LaTeX{}u je doporučeno použít jako šablonu pro text skutečné
  závěrečné práce studenta.}

\annotation[english]{Sample text of thesis at the \kitextdepten,
  \kitextfacultyen, \kitextuniven{} and, at the same time,
  documentation of the \LaTeX{} style for the text. The source text in
  \LaTeX{} is recommended to be used as a template for real student's
  thesis text.}

%% Klíčová slova práce, včetně anglických. Oddělená (obvykle) středníkem.
\keywords{styl textu; závěrečná práce; dokumentace; ukázkový text}
\keywords[english]{text style; thesis; documentation; sample text}

%% Volitelná specifikace příloh textu práce, i anglicky. Výchozí je
%% 'elektronická data v systému katedry informatiky / electronic data
%% in system of department of computer science'.
%\supplements{nejlepší software všech dob}
%\supplements[english]{the best software of all times}

%% Volitelné poděkování. Stručné! Výchozí je prázdné. Makro nemá
%% nepovinný parametr pro uvedení jazyka.
\thanks{Děkuji, děkuji, děkuji.}

%% Cesta k souboru s bibliografií pro její sazbu pomocí BibLaTeXu
%% (zvolenou nepovinným parametrem biblatex makra
%% \documentclass). Použijte pouze při této sazbě, ne při (výchozí)
%% sazbě v prostředí thebibliography.
\bibliography{bibliografie.bib}

%% Další dodatečné styly (balíky) potřebné pro sazbu vlastního textu
%% práce.
\usepackage{lipsum}
\usepackage{longtable}

\begin{document}
%% Sazba úvodních stran -- titulní, s bibliografickými údaji, s
%% anotací a klíčovými slovy, s poděkováním a prohlášením, s obsahem a
%% se seznamy obrázků, tabulek, vět a zdrojových kódů (pokud jejich
%% sazba není vypnutá).
\maketitle

%% Vlastní text závěrečné práce. Pro povinné závěry, před přílohami,
%% použijte prostředí kiconclusions. Povinná je i příloha s obsahem
%% elektronických dat.

%% -------------------------------------------------------------------

\newcommand{\BibLaTeX}{\textsc{Bib}\LaTeX}

% \noindent\textcolor{red}{\LARGE Upozornění: Následující text
%   dokumentace stylu, vyjma přílohy~\ref{sec:ObsahData}, je rozpracovaná
%   a (značně) neúplná verze!!!}

\section{Úvod}
Slepice nebo vejce? Spor starý jak lidstvo samo. Jistota je, že vejce jsou základní potravina. Bohužel je zde opravdu vidět rozdíl v kvalitě a ceně. Z tohoto důvodu se za posledních několik let rozmohly domácí chovy nosnic. Ačkoliv se na první pohled může zdát, že to není nic složitého, opak je pravdou. Proto je na místě využít moderních technologií a některé činnosti si usnadnit.
\section{popis chovu drůbeže}
•	životní nároky
•	co hlídat
•	jak se o to starat
•	způsoby chovu
•	snášení vajec
•	vliv změny délky dne a noci

\section{komponenty automatického kurníku}

\subsection{řídící jednotka}
\subsubsection{arduino mega}
\subsubsection{wifi shield}
\subsubsection{rtc modul}
\subsubsection{kamera modul}

\subsection{zdroje napětí}
\subsubsection{12v větev}
\subsubsection{5v větev}
\subsubsection{solární napájení}

\subsection{způsoby zavírání}
\subsubsection{ovládání časem / čipy}
\subsection{způsoby krmení}
\subsubsection{časově}
\subsubsection{voda}
\subsubsection{zrno}

\subsection{simulace delšího dne}
\subsubsection{časem}
\subsubsection{fotodiodou}

\subsection{zabezpečení proti dravcům a hlodavcům}
\subsubsection{zámek dvířek}
\subsubsection{zavírání, když všichni venku}
\subsubsection{pokličky}

\subsection{měření prostředí}
\subsubsection{teplota}
\subsubsection{vlhkost}

\subsection{lokální display}
\subsection{počítání slepic}
\subsubsection{využití nfc čipu (experimentálně)}
\subsection{rozpoznání obrazu}
\subsubsection{hardware}
\subsubsection{software}
\subsubsection{???}


\section{lokální ovládání kurníku}
\subsection{display}
•	opakování aktuálních informací (teplota, vlhkost, čas, možná odpočet do další akce)
•	výpis automaticky probíhající akce + delay
•	výpis manuálně spuštěné akce + delay

\subsection{klávesnice}
•	otevřít / zavřít
•	rozsvítit / zhasnout
•	krmení / voda
\section{vzdálené ovládání }
\subsection{web server}


\subsection{databáze a ukládání dat}
\subsection{webová aplikace}
\subsubsection{změna nastavení časů pro funkce}
\subsubsection{čtení naměřených hodnot}
\subsubsection{ruční ovládání akcí}

\subsection{notifikace}

\section{schéma zapojení}
\subsection{externí prvky}
•	technické zpracování dvířek
•	zpracování krmidla
\subsection{elektronické prvky}
•	schéma zapojení

\section{analýza dat}
\subsection{}

%% Závěry práce. V jazyce práce a anglicky. Text pro jiný než
%% nastavený jazyk práce (nepovinným parametrem language makra
%% \documentclass, výchozí český) se zadává použitím makra s uvedením
%% jazyka jako nepovinného parametru.
\begin{kiconclusions}
Závěr práce v \uv{českém} jazyce.
\end{kiconclusions}

\begin{kiconclusions}[english]
Thesis conclusions in \uv{English}.
\end{kiconclusions}

%% Přílohy obsahu textu práce, za makrem \appendix.
\appendix

\section{První příloha}
Text první přílohy

\section{Druhá příloha}
Text druhé přílohy

%% Obsah elektronických dat. Poslední příloha. Upravte podle vlastní
%% práce!
\section{Obsah elektronických dat} \label{sec:ObsahData}

Na samotném konci textu práce je uveden stručný popis obsahu
elektronických dat odevzdaných v systému katedry informatiky spolu s
textem. Tato data jsou nedílnou součástí práce a tvoří (datovou)
přílohu textu práce. Povinné položky struktury dat jsou:

\begin{description}

\item[\texttt{text/}] \hfill \\
  Adresář s textem práce ve formátu PDF, vytvořený s~použitím
  závazného stylu KI PřF UP v~Olomouci pro závěrečné práce, včetně
  všech (textových) příloh, a~všechny soubory potřebné pro
  bezproblémové vytvoření PDF dokumentu textu (případně v~ZIP
  archivu), tj.~zdrojový text textu a příloh, vložené obrázky, apod.

\item[\texttt{README.*}] \hfill \\
  Textový soubor (s příponou např. \texttt{.txt}) s informacemi o
  opakovatelném způsobu použití ostatních dat práce -- typicky plně
  reprodukovatelný co nejúplnější funkční postup zprovoznění software
  vytvořeného v~rámci práce, tzn. jeho případné instalace/nasazení a
  spuštění, včetně uvedení všech požadavků pro bezproblémový provoz;
  za zprovoznění software se nepovažuje zpřístupnění (např. po
  Internetu) již někde zprovozněného software.

\item[\texttt{*}] \hfill \\
  Adresáře a soubory s veškerými ostatními autorskými daty práce
  (případně v~ZIP archivu) -- typicky spustitelné a další soubory
  software vytvořeného v rámci práce potřebné pro bezproblémový provoz
  software, případně jeho instalační program, a kompletní zdrojové
  texty software a další data nutná pro plně reprodukovatelné korektní
  vytvoření spustitelných souborů.

\end{description}

Dále mohou data obsahovat například:

\begin{itemize}

%\item[\texttt{data/}] \hfill \\
\item
  ukázková a~testovací data použitá v~práci nebo pro potřeby posouzení
  práce v rámci její obhajoby,

%\item[\texttt{literature/}] \hfill \\
\item
  položky bibliografie v elektronické podobě, příp.~jiná relevantní literatura
  a dokumentace vztahující se k~práci,

%\item[\texttt{install/}] \hfill \\
\item
  cizí data (software) potřebná pro bezproblémové použití autorských
  dat práce (software), která nejsou standardní součástí
  předpokládaného (softwarového) vybavení uživatele.

\end{itemize}

U~veškerých cizích obsažených materiálů jejich
zahrnutí dovolují podmínky pro jejich veřejné šíření nebo přiložený souhlas
držitele práv k užití. Pro všechny použité (a~citované) materiály,
u~kterých toto není splněno a~nejsou tak obsaženy, je uveden
jejich zdroj, např.~webová adresa, v~bibliografii nebo textu práce
nebo souboru \texttt{README.*}.

%% -------------------------------------------------------------------

%% Sazba volitelného seznamu zkratek, za přílohami.
\printglossary

%% Sazba povinné bibliografie, za přílohami (případně i za seznamem
%% zkratek). Při použití BibLaTeXu použijte makro
%% \printbibliography. jinak prostředí thebibliography. Ne obojí!

%% Sazba i v textu necitovaných zdrojů, při použití
%% BibLaTeXu. Volitelné.
\nocite{*}
%% Vlastní sazba bibliografie při použití BibLaTeXu.
\printbibliography

%% Bibliografie, včetně sazby, při NEpoužití BibLaTeXu.
% \begin{thebibliography}{9}
%\bibitem{kniha2} \uppercase{Hawke}, Paul. NanoHttpd: Light-weight HTTP server designed for embedding in other applications. GitHub [online]. 2014-05-12. [cit. 2014-12-06]. Dostupné z: \url{https://github.com/NanoHttpd/nanohttpd}
%
%\bibitem{jeske13} \uppercase{Jeske}, David; \uppercase{Novák}, Josef. Simple HTTP Server in \csharp: Threaded synchronous HTTP Server abstract class, to respond to HTTP requests. CodeProject: For those who code [online]. 2014-05-24. [cit. 2014-12-06]. Dostupné z: \url{http://www.codeproject.com/Articles/137979/Simple-HTTP-Server-in-C}
%
%\bibitem{uzis2012} \uppercase{ÚSTAV ZDRAVOTNICKÝCH INFORMACÍ A STATISTIKY ČR}. Lékaři, zubní lékaři a farmaceuti 2012 [online]. Praha 2, Palackého náměstí 4: Ústav zdravotnických informací a statistiky ČR, 2012 [cit. 2014-12-06]. ISBN 978-80-7472-089-5. Dostupné z: \url{http://www.uzis.cz/publikace/lekari-zubni-lekari-farmaceuti-2012}
% \end{thebibliography}

%% Sazba volitelného rejstříku, za bibliografií.
\printindex

\end{document}

%%% Local Variables:
%%% mode: latex
%%% TeX-master: t
%%% End:
